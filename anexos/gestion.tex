\chapter{Planificación y gestión del proyecto}

El Apéndice \thechapter\  puede incluirse como recomendación si se ha realizado un proyecto de tipo práctico práctico de desarrollo del software.

De forma general se puede incluir la especificación requerimientos del proyecto, la información sobre stakeholders del proyecto, la justificación (interés, oportunidad, coste/beneficio…), etc.

En cuanto a la planificación: se indicarán las diferentes tareas a desarrollar durante el proyecto ordenadas cronológicamente.

En cuanto a la descripción: breve definición de las tareas indicando su relación con los requisitos del proyecto, método de trabajo, tecnologías a utilizar para desarrollar cada tarea, posibles obstáculos y riesgos.

En cuanto a la validación: qué método se utilizará para validar el proyecto (si se trata de desarrollo de software, indicando métricas referidas a "code coverage", etc., pruebas unitarias, etc.).

En cuanto al estudio de costes y sostenibilidad: se indicarán las horas de programador necesarias para llevarlo a cabo, gasto en recursos, bibliografía, tiempos de conexión, etc. 

En cuanto a la temporización, lo suyo es un Diagrama de Gantt donde quede claro el tiempo invertido en la realización de cada una de las tareas y su secuenciación/entrelazamiento a los largo de los meses de relización del proyecto.

Finalmente se acabará justificando la sostenibilidad de los productos software generados en cuanto a costes de prueba y mantenimiento del software.

Una posible estructuración  de la gestión y planificación del proyecto podría ser en las siguientes subsecciones:

\section{Metodología/Ciclo de vida} 

\section{Herramientas y plataformas utilizadas para el desarrollo del proyecto}

\section{Costes del proyecto}

\section{Gestión del proyecto. Planificación temporal. Diagrama de Gantt}


