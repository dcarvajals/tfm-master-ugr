\section{Motivación}\label{section:motivacion}

Durante el tiempo que llevo desarrollando software para diversos clientes, he identificado una creciente necesidad de herramientas que reduzcan el tiempo dedicado al análisis de requisitos de software. Esta tarea es una de las más complejas, ya que requiere comprender con precisión las necesidades del cliente para mejorar sus procesos. Aunque los desarrolladores tenemos un dominio claro de las tecnologías disponibles para crear diversos productos, cada cliente presenta necesidades únicas. Sería altamente beneficioso contar con herramientas que permitan automatizar el desarrollo de software a partir de requisitos redactados en lenguaje natural, proporcionando una base sólida sobre la cual construir el resto del proyecto.

Personalmente, también me interesa mucho desarrollar herramientas que agilicen la creación de otros productos. Es interesante observar cómo se pueden automatizar procesos que los programadores realizamos manualmente o de memoria. Considerando el impacto global actual de la IA, se propone aprovechar esta tecnología para mejorar la eficiencia y productividad de los programadores. Utilizar IA para optimizar nuestro trabajo no solo mejorará los resultados, sino que también permitirá a los desarrolladores enfocarse en tareas más estratégicas y creativas.