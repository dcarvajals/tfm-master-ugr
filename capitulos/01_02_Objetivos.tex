\section{Objetivos}\label{section:objetivos}

Los objetivos de este trabajo se formularon teniendo en cuenta el tiempo, los recursos y la tecnología disponibles.

\subsection{Objetivo general}

Extender y mejorar la herramienta TDDT4IoTS \textit{(Test-Driven Development Tool for IoT-based Systems)} mediante la integración de técnicas y tecnologías de \textbf{inteligencia artificial}, con el fin de automatizar ciertas tareas en el desarrollo de sistemas IoT, concretamente las relativas a modelado, para mejorar la eficiencia y minimizar errores humanos.

\subsection{Objetivos especificos}

\begin{enumerate}
	\item Análizar técnicas de IA a aplicar: Investigar y evaluar las técnicas y tecnologías de \textbf{inteligencia artificial} más adecuadas para integrarlas en la herramienta TDDT4IoTS, enfocadas en el análisis de información textual referente a la especificación de sistemas IoT.
	
	\item Automatizar el análisis de las especificaciones del sistema: Implementar modelos de IA que permitan analizar automáticamente descripciones de sistemas especificadas en forma de casos de uso extendidos, identificando elementos clave, como posibles clases, atributos y relaciones.
	
	\item Generar automáticamente diagramas UML: Crear un módulo dentro de TDDT4IoTS que, a partir del análisis de casos extendidos de uso, genere automáticamente diagramas de clases conceptuales, que luego puedan refinarse para obtener diagramas de clases en UML del dominio del diseño de la solución.

\end{enumerate}