\section{Estructura del trabajo}\label{section:estructura}

El resto de la memoria de este Trabajo Fin de Máster (TFM) está organizada de la siguiente manera:

\textbf{Capítulo 2}: Está compuesto por varias secciones que abordan diferentes aspectos de los modelos de lenguaje grandes o LLM \textit{(Large Language Models)} y ajuste fino, la generación automática de UML, la integración de software con IA. Comienza con una discusión sobre los \textbf{modelos de lenguaje} y NLP, incluyendo modelos base de OpenAI y modelos para procesamiento de texto. Luego, profundiza en el ajuste fino de modelos, con subsecciones que detallan ajustes específicos para tareas, dominios, adversarial, entre otros. También se exploran las técnicas de generación automática de diagramas UML y la integración del software con IA, abordando temas como optimización y portabilidad de modelos y mapas sistemáticos de técnicas. Finalmente, se discute el desarrollo de nuevos productos con IA, proporcionando una visión amplia y detallada de las tecnologías y metodologías actuales en el campo de la IA.

\textbf{Capítulo 3}:  Abarca el desarrollo del trabajo, comenzando con un análisis de requisitos que incluye la descripción general, los requisitos funcionales y no funcionales, así como restricciones y recursos utilizados. Luego, aborda la arquitectura del sistema, cubriendo las capas del cliente, conexión, negocio, nube y datos. A continuación, se describe el proceso de desarrollo e implementación, estructurado en cinco fases: análisis de requerimientos, diseño, configuración, desarrollo e integración de componentes. Finalmente, se presenta una metodología para el desarrollo de la aplicación web, destacando las interfaces y la evaluación de la extensión dentro de la aplicación.

\textbf{Capítulo 4}: Presenta los resultados obtenidos a partir de una encuesta realizada a 15 estudiantes que utilizaron la aplicación mejorada. Los resultados se analizaron y se visualizaron mediante gráficos, abordando temas como la usabilidad de la interfaz, la precisión de los diagramas generados, el tiempo de generación de los mismos y el porcentaje de modificaciones necesarias. A partir de estos datos, se ofrecen recomendaciones para mejorar la aplicación, evaluando su desempeño en el contexto del desarrollo de sistemas IoT.

\textbf{Capítulo 5:} Este capítulo concluye que los objetivos planteados al inicio del proyecto fueron cumplidos satisfactoriamente. Se destaca que la aplicación ayudo en el desarrollo de sistemas IoT y permitió una mejor comprensión y generación de diagramas de clases a partir de descripciones de casos de uso. Además, se discuten posibles líneas de mejora para futuras versiones del sistema y su potencial aplicabilidad en otros contextos de IoT.