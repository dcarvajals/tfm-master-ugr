\section{Estructura del trabajo}\label{section:estructura}

El resto de la memoria está organizada de la siguiente manera:

\textbf{Capítulo 2}: está compuesto por varias secciones que abordan diferentes aspectos de los modelos de lenguaje y ajuste fino, la generación automática de UML, la integración de software con IA, y la generación automática de código y prototipos. Comienza con una discusión sobre los modelos de lenguaje y NLP, incluyendo modelos base de OpenAI y modelos para procesamiento de texto. Luego, profundiza en el ajuste fino de modelos, con subsecciones que detallan ajustes específicos para tareas, dominios, adversarial, entre otros. También se exploran las técnicas de generación automática de UML y la integración del software con IA, abordando temas como optimización y portabilidad de modelos, mapas sistemáticos de técnicas, y análisis de fallos en redes de software. Finalmente, se discute el desarrollo de nuevos productos con IA, proporcionando una visión amplia y detallada de las tecnologías y metodologías actuales en el campo.
