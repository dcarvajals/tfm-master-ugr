\label{chapter:introduccion}\chapter{Introducción}

El desarrollo de sistemas IoT ha crecido exponencialmente en los últimos años, lo que ha generado una demanda creciente de herramientas que automaticen y optimicen el proceso de diseño y desarrollo de estos sistemas. A medida que las soluciones IoT se vuelven más complejas, la necesidad de minimizar errores humanos y acortar los tiempos de desarrollo es más crítica. Las herramientas CASE (\textit{Computer-Aided Software Engineering})  juegan un papel clave en la mejora de la productividad y en la reducción de errores mediante la automatización de tareas rutinarias.

En este contexto, la inteligencia artificial (IA) emerge como una tecnología estratégica para optimizar el desarrollo de sistemas IoT. La capacidad de la IA para procesar grandes volúmenes de datos y aprender de ellos puede ser utilizada para automatizar tareas como el análisis de requisitos y la generación de diagramas UML a partir de descripciones textuales. Este enfoque no solo mejora la eficiencia del proceso de desarrollo, mitigando la posible comisión de errores humanos, sino que también ayuda a los desarrolladores a concentrarse en tareas más creativas y/o complejas.

El presente trabajo tiene como objetivo principal extender y mejorar la herramienta  TDDT4IoTS, integrando técnicas de IA que permitan automatizar tareas clave en el desarrollo de sistemas IoT. A través de la implementación de modelos preentrenados de IA, se pretende mejorar la precisión en el análisis de especificaciones de sistemas y la generación automática de diagramas de clases a partir de dichas especificaciones.

A lo largo de este trabajo se describen los conceptos clave y avances relacionados con la integración de IA en herramientas CASE para el desarrollo de sistemas IoT. En primer lugar, se presenta un análisis del estado del arte de las tecnologías de IA, con especial énfasis en el uso de modelos preentrenados y técnicas de procesamiento de lenguaje natural o NLP (\textit{Natural Language Processing}). Luego, se detalla el proceso de desarrollo e implementación de la extensión de TDDT4IoTS, abordando los requisitos, la arquitectura del sistema y la integración de los modelos de IA. Finalmente, se exponen los resultados obtenidos a partir de pruebas realizadas con usuarios y se discuten las conclusiones más relevantes del proyecto.


\section{Motivación}\label{section:motivacion}

Aquí hay que escribir una motivación personal, pero también una motivación de tu trabajo en el área. Te tienes que plantear por qué es interesante tu trabajo en el campo.

\section{Objetivos}\label{section:objetivos}

Tanto una descripción de los objetivos generales, como una lista de objetivos específicos que, posteriormente, deben cubrirse en el capítulo de \nameref{chapter:conclusiones} (Capítulo~\ref{chapter:conclusiones}).

\section{Estructura del trabajo}\label{section:estructura}

Aquí indicarás cual es la estructura del trabajo. Por ejemplo, en el capítulo~\ref{chapter:estado-arte} (\nameref{chapter:estado-arte}) se describirán los trabajos relacionados con este TFM.




