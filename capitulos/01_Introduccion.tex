\label{chapter:introduccion}\chapter{Introducción}

La introducción de un Trabajo Fin de Máster (TFM) es una sección fundamental que sirve como presentación y puerta de entrada a la investigación realizada. Su principal objetivo es proporcionar al lector una visión general clara y concisa del tema que se va a abordar, despertando su interés y motivación para continuar leyendo. En esta sección, se debe exponer el contexto y la relevancia del tema elegido, justificando así la importancia y la necesidad de la investigación realizada. Además, se deben presentar los objetivos y la estructura del trabajo, guiando al lector a través de los capítulos y secciones que se van a desarrollar. Una buena introducción debe ser capaz de captar la atención del lector, proporcionando una idea clara y precisa de lo que se va a tratar en el TFM, y dejando una primera impresión positiva y duradera.

\begin{enumerate}

\item Concisión y claridad: La introducción debe ser concisa y clara. Evita divagar o incluir información innecesaria. Ve directo al punto y asegúrate de que el lector pueda comprender fácilmente el tema y la importancia de tu investigación.

\item Contexto y antecedentes: Proporciona un contexto adecuado para tu tema. Explica brevemente los antecedentes históricos, teóricos o prácticos relevantes para tu investigación. Esto ayudará al lector a comprender la evolución y la importancia del tema elegido.

\item Relevancia e importancia: Establece claramente la relevancia y la importancia de tu tema. Explica por qué es significativo y por qué vale la pena investigar. Demuestra que entiendes el impacto potencial de tu trabajo y cómo contribuye al campo de estudio.

\item Relevancia e importancia: Establece claramente la relevancia y la importancia de tu tema. Explica por qué es significativo y por qué vale la pena investigar. Demuestra que entiendes el impacto potencial de tu trabajo y cómo contribuye al campo de estudio.

\item Objetivos y preguntas de investigación: Presenta de manera clara y precisa los objetivos de tu TFM. Indica qué preguntas o hipótesis estás tratando de responder o abordar. Asegúrate de que los objetivos sean específicos, medibles, alcanzables, relevantes y limitados en el tiempo (SMART).

\item Estructura del trabajo: Proporciona una descripción general de la estructura de tu TFM. Guía al lector a través de los capítulos o secciones principales, explicando brevemente el contenido y la finalidad de cada parte. Esto ayuda al lector a comprender la organización lógica de tu trabajo.

\end{enumerate}

Estilo y tono: Utiliza un estilo de escritura claro, sencillo y accesible. Mantén un tono formal y académico, pero evita un lenguaje excesivamente

Estas subsecciones pueden ir dentro del fichero o fuera. Normalmente solo deberías sacar las subsecciones que ocupen más espacio en tu texto.

\section{Motivación}\label{section:motivacion}

Aquí hay que escribir una motivación personal, pero también una motivación de tu trabajo en el área. Te tienes que plantear por qué es interesante tu trabajo en el campo.

\section{Objetivos}\label{section:objetivos}

Los objetivos de este trabajo se formularon teniendo en cuenta el tiempo, los recursos y la tecnología disponibles.

\subsection{Objetivo general}

Extender y mejorar la herramienta TDDT4IoTS (Test-Driven Development Tool for IoT-based Systems) mediante la integración de técnicas y tecnologías de inteligencia artificial, con el fin de automatizar tareas en el desarrollo de sistemas IoT, mejorar la eficiencia y minimizar errores humanos.

\subsection{Objetivos especificos}

\begin{enumerate}
	\item Análisis de IA: Investigar y evaluar las técnicas y tecnologías de inteligencia artificial más adecuadas para integrar en la herramienta TDDT4IoTS, enfocadas en el análisis de información textual referente a la especificación de sistemas IoT.
	
	\item Automatización del Análisis: Desarrollar e implementar algoritmos de IA que permitan analizar automáticamente descripciones de sistemas especificadas en forma de casos de uso extendidos, identificando elementos clave como posibles clases, atributos y relaciones.
	
	\item Generación de Diagramas: Crear un módulo dentro de TDDT4IoTS que, a partir del análisis de casos de uso, genere automáticamente diagramas de clases conceptuales, que luego puedan refinarse para obtener diagramas de clases del dominio del diseño de la solución.

\end{enumerate}

\section{Estructura del trabajo}\label{section:estructura}

El resto de la memoria está organizada de la siguiente manera:

\textbf{Capítulo 2}: está compuesto por varias secciones que abordan diferentes aspectos de los modelos de lenguaje y ajuste fino, la generación automática de UML, la integración de software con IA, y la generación automática de código y prototipos. Comienza con una discusión sobre los modelos de lenguaje y NLP, incluyendo modelos base de OpenAI y modelos para procesamiento de texto. Luego, profundiza en el ajuste fino de modelos, con subsecciones que detallan ajustes específicos para tareas, dominios, adversarial, entre otros. También se exploran las técnicas de generación automática de UML y la integración del software con IA, abordando temas como optimización y portabilidad de modelos, mapas sistemáticos de técnicas, y análisis de fallos en redes de software. Finalmente, se discute el desarrollo de nuevos productos con IA, proporcionando una visión amplia y detallada de las tecnologías y metodologías actuales en el campo.




