\label{chapter:estado-arte}\chapter[Estado del arte]{Estado del arte / Trabajos previos}

Todos los trabajos tendrán el formato de IEEEtran de referencia siguiente: \url{https://www.bibtex.com/s/bibliography-style-base-ieeetr/}.

El estilo de bibliografía IEEEtran es uno de los estilos de cita y referencia más comúnmente utilizados en las áreas de ingeniería, informática y tecnología. Este estilo ha sido estandarizado por el Instituto de Ingenieros Eléctricos y Electrónicos (IEEE, por sus siglas en inglés) y se utiliza ampliamente en publicaciones académicas y conferencias en estas disciplinas. A continuación, se describe una descripción general del estilo de bibliografía IEEEtran:

\section{Formato general}

El estilo IEEEtran utiliza un sistema de numeración en el texto para las citas, lo que significa que las citas se indican mediante números entre corchetes, por ejemplo: [1], [2,3], [4], etc.

Las referencias bibliográficas se enumeran en orden numérico al final del documento en una sección titulada "Referencias".

Cada referencia se identifica en el texto por un número único entre corchetes, que corresponde al número de orden en la lista de referencias.

\section{Estructura de las referencias}

Algunas consideraciones acerca de la estructura de las referencias son:

\begin{itemize}
\item Autoría: El estilo IEEEtran sigue el formato autor, título, publicación, fecha. La inicial de los nombres se escriben primero, seguido de los apellidos de los autores, separados por comas. Si hay más de un autor, se separan por comas, y el último autor se conecta con la conjunción ``and''. Por ejemplo: F. López, M. Pérez, and J. Sánchez. Salvo que el número de autores sean 6 o más, en cuyo caso no se añade el ``and''.
\item Año de publicación: El año de publicación se indica al final con un punto: 2021.
\item Título: El título de la obra se escribe entre comillas.
\item Fuente de publicación (siempre en cursiva, salvo enlaces): Dependiendo del tipo de fuente, se incluyen diferentes elementos.

    \begin{itemize}
    \item Para artículos de revistas: Se indica el nombre de la revista en letra cursiva, seguido del número de volumen y el número de la publicación (vol. 103, no. 9) y los números de página del artículo (pp. 820--827). A veces, las revistas no tienen números o volúmenes, en cuyo caso se omite.
    \item Para libros: Se escribe el título del libro en cursiva y la editorial.
    \item Para capítulos de libros y congresos: Se indica el título del capítulo/artículo, seguido de la frase ``in Proceedings of'' y luego el título del libro/congreso (en cursiva) y el número de páginas del capítulo (pp. 820--827).
    \end{itemize}
\end{itemize}

\section{Otros detalles}

Las referencias se ordenan por orden de aparición en el texto. Si una obra no tiene autor identificado, se cita el título de la obra en lugar del autor.

Puedes leer más en la documentación oficial del paquete:
\url{https://ctan.fisiquimicamente.com/macros/latex/contrib/IEEEtran/IEEEtran_HOWTO.pdf}

\section{Descripción del capítulo}

En este capítulo se narrarán los trabajos relacionados con este TFM, trabajos previos de otros autores, con citas. Por ejemplo, muchos aspirantes en escuelas de negocios encuentran difícil escribir artículos de investigación, y pocos reciben un curso específico sobre cómo presentar su trabajo de investigación en un formato escrito. Sin embargo, la publicación es a menudo crucial para el avance de la carrera y la investigación para obtener becas, calificaciones académicas o todas estas motivaciones. El-Chaaran y otros\cite{el2021write} llaman a estimular la motivación por la investigación y satisfacer la urgencia de investigar entre los estudiantes de escuelas de negocios. Otros autores como Evans\cite{evans2011write} o Duke\cite{duke2018postgraduate} también tienen trabajos parecidos en otras áreas. Pero si quieres un libro, puedes revisar el trabajo de Wiese\cite{wiese2023write}. Recuerda que puedes citar varios autores, varias veces, según te convenga\cite{el2021write, evans2011write,duke2018postgraduate} y que debes evitar referencias huérfanas\cite{autordesconocido} que aparecen siempre con una interrogación (?).

Así se referencia un artículo de revista como el de El-Chaarani\cite{el2021write}, así se cita un libro como el mítico \emph{Introduction to Computer Graphics} de Foley\cite{foley1994introduction}, así se cita un congreso como el artículo de Schwarz\cite{schwarz2007modeling}, y así se cita una web como la Wikipedia\cite{wikipediaComputerGraphics}. En la aplicación getBibTex\cite{getbibtexBibTeXGenerator} puedes preparar tus citas Web y referenciarlas: \url{https://www.getbibtex.com/}.

