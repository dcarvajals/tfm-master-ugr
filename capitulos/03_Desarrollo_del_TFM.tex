\chapter{Desarrollo del TFM}\label{chapter:desarrollo}

En esta parte describirás el trabajo desarrollado, y utilizarás las secciones que creas conveniente para explicarlo.

\section{Cómo colocar tablas y figuras}

Por ejemplo, en esta sección se explica como colocar tablas y figuras en \LaTeX{}. Al escribir un documento en \LaTeX{}, es importante saber cómo posicionar correctamente imágenes y tablas. \LaTeX{} ofrece varias opciones para controlar la posición de estos elementos en la página. Por ejemplo, se puede utilizar las opciones ``h'', ``t'', ``b'', ``p'', ``!'' para indicar que la imagen debe ser colocada "aquí" (\emph{here}), en la parte superior (\emph{top}) o inferior (\emph{bottom}) de la página, en una página separada (\emph{page}) o en la posición exacta especificada (!), respectivamente.

Tanto las imágenes como las tablas son consideradas como "objetos flotantes" en \LaTeX{}. Esto significa que \LaTeX{} intentará posicionarlos en la página de la mejor manera posible, teniendo en cuenta el flujo del texto y el espacio disponible. En algunos casos, esto puede resultar en un posicionamiento inesperado de los objetos flotantes.

Para tener un mayor control sobre el flotado de imágenes y tablas, se pueden utilizar los paquetes \textbf{\textbackslash usepackage\{float\}} y \textbf{\textbackslash usepackage\{caption\}}. Estos paquetes ofrecen opciones adicionales para controlar la posición y el comportamiento de los objetos flotantes. Por ejemplo, se puede utilizar la opción ``H'' para forzar a \LaTeX{} a colocar el objeto flotante exactamente en la posición especificada, sin ajustes. 

Revisa esta web para más información: \url{https://www.overleaf.com/learn/latex/Positioning_images_and_tables}.
