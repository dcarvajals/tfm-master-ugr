\chapter{Conclusiones y trabajo futuro}\label{chapter:conclusiones}

En este capítulo se presentan las principales conclusiones derivadas del desarrollo de este trabajo, destacando los resultados obtenidos y su impacto en el ámbito del desarrollo de sistemas IoT. Además, se discuten las posibles líneas de investigación futuras y mejoras que pueden implementarse en la herramienta desarrollada, con el fin de incrementar su funcionalidad a contextos más complejos.

\section{Principales conclusiones del trabajo desarrollado}

El objetivo de extender y mejorar la herramienta TDDT4IoTS mediante la integración de inteligencia artificial se ha cumplido con éxito. La implementación de modelos preentrenados, como los ofrecidos por OpenAI, permitió mejorar el desarrollo de tareas críticas al momento de crear sistemas IoT, mejorando la reducción de errores humanos. La evaluación con los estudiantes mostró que los diagramas generados tuvieron una buena aceptación, con tiempos de generación razonables y una alta satisfacción respecto a la usabilidad de la interfaz, lo que indica que las mejoras implementadas e incorporadas a la herramienta se vieron reflejadas en el nivel de satisfacción de los usuarios que han usado dicha herramienta.

El análisis de las tecnologías de inteligencia artificial disponibles concluyó que los modelos preentrenados de OpenAI eran uno de los más adecuados para ser integrados en TDDT4IoTS, dado su flexibilidad y tiempos de entrenamiento reducidos. A pesar de que se consideró inicialmente desarrollar un modelo desde cero, la alternativa de los modelos preentrenados resultó ser mas adaptable, lo que permitió avanzar rápidamente en la integración de las nuevas funcionalidades implementadas en de la herramienta \textit{case} existente.

El resultado de aplicar la implementación realizada para que modelos de IA hicieran el análisis automático de descripciones textuales en forma de casos de uso fue considerablemente buena. Los modelos integrados permitieron identificar la mayor parte de elementos clave, como clases, atributos y relaciones, a partir de las especificaciones de los sistemas IoT. Los resultados reflejaron que los estudiantes encontraron útiles los diagramas generados, siendo necesario aplicar solo pequeñas modificaciones a dichos diagramas, lo que confirma que la automatización del análisis de las especificaciones como entrada a la herramienta cumplió con los objetivos propuestos.

El módulo de generación automática de diagramas de clases demostró ser correcto, ya que los diagramas generados fueron cercanos a la solución final. Si bien algunos usuarios indicaron la necesidad de realizar ajustes manuales menores, esto es normal en procesos de diseño asistido. Los tiempos de generación, evaluados por los estudiantes, fueron en su mayoría satisfactorios, con tiempos entre 1 y 3 minutos, lo que cumple con las expectativas de un proceso automatizado dentro del desarrollo de sistemas IoT.

Finalmente la herramienta TDDT4IoTS ha sido significativamente mejorada mediante la integración de modelos de IA, logrando un avance importante en la automatización de tareas que anteriormente requerían intervención manual. Esto no solo mejora la eficiencia del proceso de desarrollo de sistemas IoT, sino que también trata de minimizar los errores humanos, aumentando la productividad de los ingenieros de software que desarrollen sistemas IoT. Con todo ello, se ha conseguido mejorar la herramienta, proporcionando una solución robusta para los desarrolladores de este tipo de sistemas.

\section{Trabajo futuro}

Para continuar con la línea de investigación desarrollada en este trabajo, me complace comunicar que seguiré mis estudios de doctorado en la Universidad de Granada. En este contexto, planeo explorar diversos aspectos que permitirán mejorar y agregar nuevas funcionalidades a la herramienta, con el objetivo de ampliar su aplicabilidad y eficacia en el ámbito del desarrollo de sistemas.

\begin{itemize}
	\item \textbf{Personalización de los prompts de entrenamiento:} Actualmente, la herramienta utiliza un prompt por defecto para entrenar los modelos. Una mejora sería permitir al usuario personalizar el prompt de entrenamiento según el modelo seleccionado. Esto ayudaría al modelo a entrenarse de manera más ajustada a las necesidades específicas del usuario (ingeniero de software), facilitando una mejor comprensión de las especificaciones del sistema IoT que se desea desarrollar.
	
	\item \textbf{Integración con documentación técnica: } Se podría mejorar la capacidad de los modelos de OpenAI para trabajar con documentación técnica más detallada, como informes o manuales de sistemas IoT. A través de la integración de estos documentos en el prompt o el uso de procesamiento en múltiples pasos, la herramienta podría generar una comprensión más profunda y precisa de los sistemas, lo que permitiría generar diagramas más adecuados a partir de la documentación.
	
	\item \textbf{Chat de ayuda para redactar casos de uso extendidos:} Una mejora futura sería integrar un chat, utilizando la API de OpenAI, que asista al usuario en la redacción de casos de uso extendidos. Esto permitiría aclarar de manera más eficiente los requisitos funcionales y no funcionales de un sistema IoT. Al ayudar al usuario a describir mejor las especificaciones del sistema, el modelo podrá comprender con mayor precisión el contexto, lo que facilitará la generación de un diagrama de clases más claro y detallado.
	
	\item \textbf{Entrenar un modelo propio en lugar de depender de OpenAI:} Desarrollar y entrenar un modelo interno desde cero, en lugar de depender de los modelos preentrenados de OpenAI. Esto permitiría un mayor control sobre el proceso de entrenamiento y la capacidad de adaptar el modelo específicamente a las necesidades de la herramienta. Para lograrlo, se podría usar Python como lenguaje principal, junto con datasets extraídos de investigaciones previas, proyectos de software existentes, y especificaciones de sistemas IoT. Esta personalización del entrenamiento mejoraría la comprensión del procesamiento de lenguaje natural (NLP) adaptado a contextos específicos, como el desarrollo de sistemas IoT, asegurando que el modelo sea más eficiente y adecuado para las necesidades del usuario.
	
\end{itemize}

Estas ideas representan solo una parte de las muchas mejoras y funcionalidades que se podrían implementar en la herramienta. Existe un gran entusiasmo por continuar trabajando en esta línea de investigación, explorando cómo la IA puede potenciar las herramientas \textit{case} y facilitar el trabajo de los desarrolladores. La integración de modelos de IA personalizados y otras innovaciones permitirán crear soluciones más eficientes y adaptadas a las necesidades del desarrollo de sistemas IoT, abriendo nuevas oportunidades para mejorar la calidad y rapidez en el diseño y construcción de software.