\chapter{Conclusiones}\label{chapter:conclusiones}

El objetivo de extender y mejorar la herramienta TDDT4IoTS mediante la integración de inteligencia artificial se ha cumplido con éxito. La implementación de modelos preentrenados, como los ofrecidos por OpenAI, permitió mejorar el desarrollo de tareas críticas al momento de crear sistemas IoT, mejorando la reducción de errores humanos. La evaluación con los estudiantes mostró que los diagramas generados tuvieron una buena aceptación, con tiempos de generación razonables y una alta satisfacción respecto a la usabilidad de la interfaz, lo que indica que las mejoras implementadas se vieron reflejadas en la herramienta.

El análisis de las tecnologías de inteligencia artificial disponibles concluyó que los modelos preentrenados de OpenAI eran uno de los más adecuados para ser integrados en TDDT4IoTS, dado su flexibilidad y tiempos de entrenamiento reducidos. A pesar de que se consideró inicialmente desarrollar un modelo desde cero, la alternativa de los modelos preentrenados resultó ser mas adaptable, lo que permitió avanzar rápidamente en la integración de la herramienta CASE.

La implementación de modelos de IA para el análisis automático de descripciones textuales en forma de casos de uso fue considerablemente buena. Los modelos integrados permitieron identificar la mayor parte de elementos clave, como clases, atributos y relaciones, a partir de las especificaciones de los sistemas IoT. Los resultados reflejaron que los estudiantes encontraron útiles los diagramas generados, con una breve necesidad de modificaciones, lo que confirma que la automatización del análisis cumplió con los objetivos propuestos.

El módulo de generación automática de diagramas de clases demostró ser correcto, ya que los diagramas generados fueron cercanos a la solución final. Si bien algunos usuarios indicaron la necesidad de realizar ajustes manuales menores, esto es normal en procesos de diseño asistido. Los tiempos de generación, evaluados por los estudiantes, fueron en su mayoría satisfactorios, con tiempos entre 1 y 3 minutos, lo que cumple con las expectativas de un proceso automatizado dentro del desarrollo de sistemas IoT.

Finalmente la herramienta TDDT4IoTS ha sido significativamente mejorada mediante la integración de modelos de IA, logrando un avance importante en la automatización de tareas que anteriormente requerían intervención manual. Esto no solo mejora la eficiencia del proceso de desarrollo de sistemas IoT, sino que también trata de minimizar los errores humanos, mejorando la herramienta y tratar de proporcionar una solución robusta para los desarrolladores de este tipo de sistemas.