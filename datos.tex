% Editar este fichero y rellenar los datos

% título y subtítulo del TFM en español
\newcommand{\tituloTFM}{Extensión de herramienta CASE para el desarrollo de sistemas IoT}
\newcommand{\subtituloTFM}{Interpretación inteligente de casos de uso}
% palabras clave en español
\newcommand{\palabrasClave}{palabra clave 1, palabra clave 2, \ldots, palabra clave N}

% título y subtítulo del TFM en inglés
\newcommand{\titleMT}{CASE tool extension for IoT systems development}
\newcommand{\subtitleMT}{Intelligent interpretation of use cases}
% palabras clave en inglés
\newcommand{\keywords}{keyword  1, keyword 2, \ldots, keyword N}

% Datos del estudiante (nombre y DNI o equivalemente)
\newcommand{\estudiante}{Dúval Carvajal Suárez}
\newcommand{\dni}{Z1655535T}

% Datos del tutor (nombre, área y departamento)
\newcommand{\tutorA}{Miguel J. Hornos Barranco}
\newcommand{\areaA}{Lenguajes y Sistemas Informáticos}
\newcommand{\departamentoA}{Lenguajes y Sistemas Informáticos}

% Datos del co-tutor (nombre, área y departamento), si no hay co-tutoror declarar vacío
\newcommand{\tutorB}{Carlos Rodríguez Domínguez}
%\newcommand{\tutorB}{}
\newcommand{\areaB}{Lenguajes y Sistemas Informáticos}
\newcommand{\departamentoB}{Lenguajes y Sistemas Informáticos}

 % poner el nombre de la imagen como logotipo o declarar vacío para ningún logotipo:
\newcommand{\logoTFM}{logoExtensiontfm}
%\newcommand{\logoTFM}{}

% versión de la memoria
\newcommand{\version}{1.0}


