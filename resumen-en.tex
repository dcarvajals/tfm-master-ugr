% Editar este fichero y rellenar con el resumen del TFM en inglés

The development of IoT systems has been increasing over time, and its implementation always presents challenges when trying to convey the central idea of the project. Like any software system, it requires a technical analysis prior to the physical development of the project. Developers tend to debate the creation of technical diagrams that allow the entire team to understand the system's full functionality. There is a CASE tool called TDDT4IOTS (Test-Driven Development Tool for IoT-based Systems), which has been enhanced by extending its functionality and equipping it with greater intelligence for developing IoT systems. The tool's primary input consists of system descriptions provided by developers in the form of extended use cases. By leveraging the OpenAI API, techniques from its models were implemented to analyze these descriptions and automatically identify key elements (such as potential classes, attributes, relationships, etc.) that should be considered in order to automatically generate a conceptual class diagram. This diagram could then be refined to produce a design-level class diagram for the solution domain. Additionally, fine-tuning has been implemented, allowing users to provide specific prompts to the model through the tool to improve the development of IoT systems in a manner that best suits their needs.
