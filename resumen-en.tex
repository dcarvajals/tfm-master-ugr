% Editar este fichero y rellenar con el resumen del TFM en inglés

In order to assist software engineers developing IoT systems by automating part of their tasks so they can work more efficiently and minimize potential human errors, it is proposed to extend the TDDT4IoTS (Test-Driven Development Tool for IoT-based Systems) tool, equipping it with greater intelligence and new capabilities to significantly improve the efficiency of the IoT systems development process, facilitating its design and implementation, and shortening time frames. To achieve this, existing Artificial Intelligence (AI) techniques and technologies that could be most suitable for extending and improving this tool will need to be analyzed in order to integrate them for the analysis of textual information regarding the specification of an IoT system. In practice, the starting point will be the system descriptions to be developed specified by the developers in the form of extended use cases, which are the inputs provided to the TDDT4IoTS tool. Through analysis using AI techniques, the key elements (possible classes, attributes, relationships, etc.) that should be considered to automatically generate a conceptual class diagram will be identified. This can then be refined to obtain a class diagram for the solution's design domain.
