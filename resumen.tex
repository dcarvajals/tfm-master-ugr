% Editar este fichero y rellenar con el resumen del TFM

El desarrollo de Sistemas basados en Internet de las Cosas IoT (\textit{Internet Of Things}) va aumentando conforme va pasando el tiempo y su trabajo tiene complicaciones al momento de concretar la idea central del proyecto. Como todo Sistema de software debe tener un análisis técnico previo al desarrollo físico del proyecto, los desarrolladores tienden a debatir sobre la creación de diagramas técnicos que permiten a todo el equipo entender el funcionamiento completo del sistema. Existe una herramienta CASE denominada TDDT4IOTS (\textit{Test-Driven Development Tool for IoT-based Systems}), la cual fue mejorada extendiendo su funcionalidad dotándola de mayor inteligencia para desarrollar sistemas IoT. La entrada principal de información por parte de la herramienta son las descripciones del sistema a desarrollar especificadas por los desarrolladores en forma de casos de uso extendidos. Utilizando la API de OpenAI se implementaron las tecnicas que brindan sus modelos para analizar estas descripciones e identificarán automáticamente los elementos clave (posibles clases, atributos, relaciones, etc.) que se deben considerar para generar, también de manera automática, un diagrama de clases conceptual, que luego podría refinarse para obtener un diagrama de clases del dominio del diseño de la solución. Además, se logro implementar el ajuste fino de forma los usuarios, mediante la herramienta, tenga la opción de poder indicarle las instrucciones necesarias al modelo que mas les interese para mejorar el desarrollo del sistema IoT.