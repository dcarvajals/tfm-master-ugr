% Editar este fichero y rellenar con el resumen del TFM

Con el fin de ayudar a los ingenieros de software que desarrollen sistemas IoT,
automatizando parte de sus tareas, para que puedan realizar su trabajo más eficientemente
y minimizar posibles errores humanos, se propone extender la herramienta TDDT4IoTS
(Test-Driven Development Tool for IoT-based Systems), dotándola de una mayor inteligencia
y de nuevas capacidades para mejorar significativamente la eficiencia del proceso de
desarrollo de sistemas IoT, facilitando su diseño e implementación y acortando tiempos.
Para ello, habrá que analizar las técnicas y tecnologías existentes de Inteligencia Artificial
(IA) que podrían ser más adecuadas para extender y mejorar esta herramienta, con el fin de
integrarlas para el análisis de información textual referente a la especificación de un sistema
IoT. En la práctica, se partirá de las descripciones del sistema a desarrollar especificadas por
los desarrolladores en forma de casos de uso extendidos, que son las entradas
proporcionadas a la herramienta TDDT4IoTS. Con el análisis mediante técnicas de IA, se
identificarán automáticamente los elementos clave (posibles clases, atributos, relaciones,
etc.) que se deben considerar para generar, también de manera automática, un diagrama
de clases conceptual, que luego podría refinarse para obtener un diagrama de clases del
dominio del diseño de la solución.
